%%%%%%%%%%%%%%%%%%%%%%%%%%%%%%%%%%%%%%%%%%%%%%%%%%%%%%%%%%%%%%%%%%%%%%%%%%%
%
% Plantilla para un artículo en LaTeX en español.
%
%%%%%%%%%%%%%%%%%%%%%%%%%%%%%%%%%%%%%%%%%%%%%%%%%%%%%%%%%%%%%%%%%%%%%%%%%%%

\documentclass{article}

% Esto es para poder escribir acentos directamente:
\usepackage[latin1]{inputenc}
% Esto es para que el LaTeX sepa que el texto está en español:
\usepackage[spanish]{babel}

% Paquetes de la AMS:
\usepackage{amsmath, amsthm, amsfonts, mathtools}
\usepackage{multicol}
\usepackage{lscape}

\DeclarePairedDelimiter{\ceil}{\lceil}{\rceil}

% Teoremas
%--------------------------------------------------------------------------
\newtheorem{thm}{Teorema}[section]
\newtheorem{cor}[thm]{Corolario}
\newtheorem{lem}[thm]{Lema}
\newtheorem{prop}[thm]{Proposición}
\theoremstyle{definition}
\newtheorem{defn}[thm]{Definición}
\theoremstyle{remark}
\newtheorem{rem}[thm]{Observación}

% Atajos.
% Se pueden definir comandos nuevos para acortar cosas que se usan
% frecuentemente. Como ejemplo, aquí se definen la R y la Z dobles que
% suelen representar a los conjuntos de números reales y enteros.
%--------------------------------------------------------------------------

\def\RR{\mathbb{R}}
\def\ZZ{\mathbb{Z}}

% De la misma forma se pueden definir comandos con argumentos. Por
% ejemplo, aquí definimos un comando para escribir el valor absoluto
% de algo más fácilmente.
%--------------------------------------------------------------------------
\newcommand{\abs}[1]{\left\vert#1\right\vert}

% Operadores.
% Los operadores nuevos deben definirse como tales para que aparezcan
% correctamente. Como ejemplo definimos en jacobiano:
%--------------------------------------------------------------------------
\DeclareMathOperator{\Jac}{Jac}

%--------------------------------------------------------------------------
\title{Diseño y análisis de algoritmos. Tarea I}
\author{Sebastián Valencia Calderón\\
	\small 201111578
}

\begin{document}
	\thispagestyle{empty}
	
	\begin{landscape}
		$$\sigma_{(firstname\ =\ "Pepito")\ \wedge\ (year\ \leq\ 45)\ \wedge\ ((motivation\ =\ "muslim")\ \vee\ (lastname\ \neq\ "Knuth"))}\ (turingaward)$$
		
		$$\Pi_{(firstname,\ motivation,\ year)}\ ((\sigma_{(firstname\ =\ "Pepito")\ \wedge\ (year\ \leq\ 45)\ \wedge\ ((motivation\ =\ "muslim")\ \vee\ (lastname\ \neq\ "Knuth"))}\ (turingaward)))$$
		
		$$origami\ \leftarrow\ \sigma_{(firstname\ =\ "Pepito")\ \wedge\ (year\ \leq\ 45)\ \wedge\ ((motivation\ =\ "muslim")\ \vee\ (lastname\ \neq\ "Knuth"))}\ (turingaward)$$
		
		$$\Pi_{(name,\ customer,\ account.number)}\ ((\sigma_{customer.sin\ =\ account.sin}\ ((cutomer\ \times\ account))))$$
		
		$$\Pi_{(name,\ customer,\ account.number)}\ ((\sigma_{customer.sin\ =\ account.sin}\ (((cutomer\ \cup\ customer)\ \times\ account))))$$
		
	\end{landscape}

\end{document}